\documentclass{article}

% Language setting
% Replace `english' with e.g. `spanish' to change the document language
\usepackage[english]{babel}

% Set page size and margins
% Replace `letterpaper' with `a4paper' for UK/EU standard size
\usepackage[letterpaper,top=2cm,bottom=2cm,left=3cm,right=3cm,marginparwidth=1.75cm]{geometry}

% Useful packages
\usepackage{amsmath}
\usepackage{graphicx}
\usepackage[colorlinks=true, allcolors=blue]{hyperref}

\title{Lab 1 Questions}
\author{Alexander Hanys}

\begin{document}
\maketitle


\section{What are the advantages and disadvantages of using the same system call interface for manipulating both files and devices?}

By using the same system call interface, there is more opportunity for code re-usability between different processes. This allows the same program to access data on both devices, and within files, instead of having to split the program into multiple programs.

\section{Would it be possible for the user to develop a new command interpreter using the system call interface provide by the operating system? How?}

yes, it could be possible to develop a new command interpreter. In order to do this, the interpreter must register functions that implement the I/O operations of the device or driver, ensuring compatability. 

\end{document}